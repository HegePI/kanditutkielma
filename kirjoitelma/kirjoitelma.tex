\documentclass[finnish,twoside,censored,essay,sw-line]{HYthesisML}

\PassOptionsToClass{openright,twoside,a4paper}{report}

\usepackage{csquotes}

\usepackage[style=numeric,bibstyle=numeric,backend=biber,natbib=true,maxbibnames=99,giveninits=true,uniquename=init]{biblatex}
\addbibresource{bibliography.bib}

\DeclareNameAlias{sortname}{family-given}

\usepackage{lmodern}         % Font package, again in some systems.
\usepackage{textcomp}        % Package for special symbols
\usepackage[pdftex]{color, graphicx} % For pdf output and jpg/png graphics
\usepackage{epsfig}
\usepackage{subfigure}
\usepackage[pdftex, plainpages=false]{hyperref} % For hyperlinks and pdf metadata
\usepackage{fancyhdr}        % For nicer page headers
\usepackage{tikz}            % For making vector graphics (hard to learn but powerful)
%\usepackage{wrapfig}        % For nice text-wrapping figures (use at own discretion)
\usepackage{amsmath, amssymb} % For better math

\singlespacing               %line spacing options; normally use single

\fussy

\title{Koneoppimisen menetelmät ja käyttökohteet lääketutkimuksessa -ja kehityksessä}

% TM: Contributors to template editors now listed in the beginning of the file in comments
\author{Heikki Pulli}

\date{\today}

\supervisors{}
\examiners{}

% \keywords{ulkoasu, tiivistelmä, lähdeluettelo}
% \additionalinformation{\translate{\track}}

% \classification{\protect{\ \\
% \  General and reference $\rightarrow$ Document types  $\rightarrow$ Surveys and overviews\  \\
% \  Applied computing  $\rightarrow$ Document management and text processing  $\rightarrow$ Document management $\rightarrow$ Text editing
% }}

\begin{document}
\maketitle

\mytableofcontents
\mainmatter

\chapter{Johdanto}





Koneoppimisen malleja on alettu käyttämään lääketutkimuksen ja -kehityksen tuotantoprosessin eri osa-alueilla ja vaiheissa. Koneoppmisen malleja voidaan hyödyntää esim. eri
solujen tunnistamiseen kuvista, kemiallisten yhdisteiden vaikutusten ennustamista eri soluihin ja parhaimmillaan jopa mallentamaan tiettyyn sairauteen tehoavaa täsmälääkettä~\cite{VamathevanJessica2019Aoml}.
On myös kuitenkin herännyt huoli datasta, jolla koulutetaan koneoppimisen mallit eri tilanteisiin. Huoli on noussut koskien datan laatua, jolla koneet koulutetaan. Tämä myös
koneoppimisen mallien käyttöönotottoa, koska koskaan ei täysin tiedetä, miten koulutettu kone on tulokseensa päätynyt.

\chapter{Lääketutkimuksessa -ja kehityksessä käytetyt koneoppimisen mallit}

\chapter{Koneoppimismallien käyttökohteet}

\chapter{Koneoppimismallien heikkoudet ja puutteet}

\cleardoublepage
\phantomsection
\addcontentsline{toc}{chapter}{\bibname}
\printbibliography

\backmatter

\end{document}

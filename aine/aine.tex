
\documentclass[finnish,twoside,censored,subject,sw-line]{HYthesisML}


\PassOptionsToClass{openright,twoside,a4paper}{report}

\usepackage{csquotes}
\usepackage[style=numeric,bibstyle=numeric,backend=biber,natbib=true,maxbibnames=99,giveninits=true,uniquename=init]{biblatex}

\addbibresource{bibliography.bib}
\DeclareNameAlias{sortname}{family-given}



\usepackage{lmodern}         % Font package, again in some systems.
\usepackage{textcomp}        % Package for special symbols
\usepackage[pdftex]{color, graphicx} % For pdf output and jpg/png graphics
\usepackage{epsfig}
\usepackage{subfigure}
\usepackage[pdftex, plainpages=false]{hyperref} % For hyperlinks and pdf metadata
\usepackage{fancyhdr}        % For nicer page headers
\usepackage{tikz}            % For making vector graphics (hard to learn but powerful)
\usepackage{amsmath, amssymb} % For better math

\singlespacing               %line spacing options; normally use single

\fussy
\title{koneoppimisen menetelmät lääkkeiden/kemiallisessa syntetisoinnin mallennuksessa}

\author{Heikki Pulli}
\date{\today}



\supervisors{}
\examiners{}


\keywords{ulkoasu, tiivistelmä, lähdeluettelo}
\additionalinformation{\translate{\track}}


\begin{document}

% Generate title page.
\maketitle

% Place ToC
\newpage
\mytableofcontents
\mainmatter

\chapter{Johdanto}

Uusien lääkkeiden tuottaminen on pitkä ja kallis prosessi. Tavallisesti aika, jossa
lääke ensin löudetään, testataan ja hyväksytään useiden eri testien läpi, vaihtelee
10-12 vuoden välillä ja hinta on 1-2 miljardin dollarin välillä.~\cite{EkinsSean2019Emlf}.

Lääkefirmat ovatkin alkaneet selvittää, kuinka eri koneoppimisen malleja voidaan
hyödyntää lääketutkimuksessa nopeuttamaan suurimpia pullonkauloja.~\cite{EkinsSean2019Emlf}.
Eri mallit voivat esimerkiksi karsia kaikista harkinnasta olevista lääkkeistä vain
lupaavimmat kandidaatit, joilla on mahdollisuus päästä testeistä läpi tuotantoon.
Tai koneoppimismalleja voidaan hyödyntää täysin uusien lääkeaineiden etsinnässä, joilla
on halutut lääkkeelliset että fysikaaliset ominaisuudet.~\cite{!lähde!}

Uuden lääkkeen löytäminen sairauteen on pitkä ja kallis prosessi.
Tähän kuuluu useita eri vaiheita ja eri vaiheet vievät eri määrän rahaa ja aikaa.
Nämä ovat sairauden aiheuttajan tunnistaminen, tähän vaikuttavan lääkkeen tunnistaminen, lääkkeen optimointi, lääkkeen ominaisuuksien analysointi ja kliiniset testit.
Näiden jälkeen lääke joko hyväksytään myyntiin tai ei.
Nämä eri vaiheet vievät tavallisesti 10 - 12 vuotta ja hintaa tälle tulee noin 2.8 miljardia dollaria.
Vaiheiden pitkän keston ja suuren hinnan takia tutkijat ja lääkefirmat ovatkin alkaneet tutkia mahdollisia keinoja, jotka nopeuttaisivat tai halventaisivat tätä lääkkeen kehityksen prosessia.

Koneoppimisenmallit ovat nousseet houkuttelevaksi vaihtoehdoksi, joka voisi nopeuttaa tätä prosessia.
viimeisen kymmenen vuoden aikana saatavilla olevan laadukkaan datan määrä on kasvanut merkittävästi ja uusia tehokkaampia koneoppimismalleja on kehitetty, joita voidaan hyödyntää lääketutkimuksessa.
Kehitetyt mallit ovatkin näyttäneet, että koneoppimismallit ovat tehokkaita työkaluja, joita voidaan hyödyntää kaikissa lääketutkimuksen prosessin vaiheissa.

Tässä tektissä paneudutaan syvemmin koneoppimismalleihin, joita käytetään uusien lääkkeiden tunnistaminseen ja näiden tunnistettujen lääkkeiden syntetisoinnin suunnitteluun.

\chapter{Uusien lääkkeiden löytäminen}

% Miksi uusien lääkkeiden löytäminen on vaikeaa?~\cite{EkinsSean2019Emlf}

Yksi ensimmäisistä lääketutkimuksen prosessin osa-aluiesta on uusien lääekyhdisteiden löytäminen joko uusiin tai jo tunnettuihin tauteihin.~\cite{EkinsSean2019Emlf}
Tämä on kuitenkin ollut tavallisesti hidas prosessi ja uuden toimivan yhdisteen löytäminen on kestänyt kahdesta kolmeen vuotta.
Lisääntynyt datan määrä on kuitenkin mahdollistanut tämän osa-alueen nopeuttamisen koneoppimismallien avulla.
Tähän ongelmaan on kehitetty useita eri koneoppimismalleja.
Mariya Popovan, Olexandr Isayev ja Alexander Tropsha tutkimusryhmä on kehittänyt mallin, joka ehdottaa uutta yhdistettä perustuen mallin syötteenä saamaan ominaisuus vektoriin.~\cite{PopovaMariya2018Drlf}
Shahar Harelin ja Kira Radinskyn tutkijaryhmä puolestaan ovat kehittäneet mallin, joka luo uusia yhdisteitä, jotka perustuvat syötteenä annettuun prototyyppi yhdisteeseen.~\cite{ShaharHarelAndKiraRadinsky}

Jotta koneoppimismalleja voidaan hyödyntää lääketutkimuksessa täytyy olla saatavilla tarpeeksi dataa tutkittavasta aiheesta.~\cite{EkinsSean2019Emlf}
Viimeisimmän kymmenen vuoden aikana saatavilla olevan datan määrä on kasvannut merkittävästi kehitettyjen tietopankkien takia.
Näitä ovat esimerkiksi PubChem ja ChEMBL.

Lääkkeitä myös kehitetään muokkaamalla jo tunnettua lääkeyhdistettä jolla on jo osittain halutut ominaisuudet.~\cite{ShaharHarelAndKiraRadinsky}


\section{virtual screening}

Erillaisten kemiallisten yhdisteiden avaruus on suuri.
On arvioitu, että erillaisia kemiallisia yhdisteitä, jotka voivat esiitnyä huoneen lämmössä ja nesteessä, voi olla välillä $10^{18} - 10^{180}$.~\cite{SotrifferChristoph2011VSPC}
Lääkkeeksi käyvien yhdisteiden määrä on taas puolestaan arvioitu olevan koko luokkaa $10^{60}$.~\cite{SotrifferChristoph2011VSPC}
Tämä itsessään esittää tarpeen tehokkaille algoritmeille ja menetelmille, jotka auttavat karsimaan tästä suuresta määrästä kemiallisia yhdisteitä vain lupaavimmat.

Virtual screening (VS) on joukko menetelmiä uusien lääkkeiden löytämiseksi.
VS menetelmillä tarkoitetaan yleisesti prosesseja, joissa käydään läpi suuria tietokantoja dataa, jotta löydetään haluttu yhdiste.~\cite{SotrifferChristoph2011VSPC}

\subsection{Koneoppimisen käyttökohteet VS:ässä}

Ongelman kuvaaminen koneoppimisongelmana~\cite{ShaharHarelAndKiraRadinsky,KadurinArtur2017dAAG}

\section{Kehitettyjä koneoppimismalleja}

\subsection{Prototyyppiin perustuva lääkkeen suunnittelu}
- Prototype based drug design~\cite{ShaharHarelAndKiraRadinsky}

\subsection{Controlled molecule generator, CMG}
- Controlled molecule generator~\cite{ShinBonggun}

CMG (Controlled molecule generator) on koneoppimismalli, joka etsii uusia molekyylejä, jotka perustuvat syötteenä annettuun molekyyliin ja joilla on ennalta määritellyt halutut ominaisuudet.~\cite{ShinBonggun}
Se eroaa muista kehitetyistä malleista siten, että se pystyy optimoimaan annetun molekyylin useampaa ominaisuutta.
Edelliset kehitetyt mallit ovat psytyneet optimoimaan vain yhtä molekyylin ominaisuutta.

CMG:n kehittäjät lähestyvät molekyylin ominaisuuksien optimointi ongelmaa merkiijonojen käännös/luonti ongelmana.
CMG opetetaan, kuinka syötteenä annettu molekyyli merkkijono käännetään molekyyli merkkijonoksi, jolla on lähimpänä haluttuja ominaisuuksia olevat omaisuudet.
CMG tulkitsee ensin annetut merkkijonot hyödyntäen DN:ää (deep network) jonka jälkeen se luo uusia molekyyli yhdisteitä hyödyntäen tätä tulkintaa ja haluttua ominaisuusvektoria.
Koska ominaisuudet annetaan vektorina, niin CMG pystyy optimoimaan useampaa ominaisuutta.
CMG lisäksi hyödyntää ennalta koulutettuja rajoite verkkoja (constraint network), jolloin vältytään luomasta mahdottomia yhdisteitä.
CMG käyttää näitä verkkoja hyödyksi käyttämällä muokattua Beam Search -algoritmia.





\chapter{Uusien lääkkeiden syntetisointi}

Yhdisteen syntetisoinnin suunnittelulla tarkoitetaan prosessia, jossa määritellään, kuinka haluttu yhdiste voidaan
tuottaa synteettisesti saatavilla olevista lähtöaineista.~\cite{ColeyConnorW2018MLiC} Retrosynteesi analyysillä
tarkoitetaan puolestaan menetelmää, jonka avulla löydetään halutun yhdisteen tuottamiseen tarvittavat lähtöaineet.
Retrosynteesi toimii siis toiseen suuntaan kuin syntetisointi. Retrosynteesissä yhdiste pilkotaan rekursiivisesti
pienempiin lähtöaineisiin kunnes jäljellä on vain saatavilla olevia lähtöaineita.

Tavallisesti yhdisteen retrosyntetisointi on vaatinut suorittavalta kemistiltä usean vuoden kokemusta ja tietoa
saatavilla olevista lähtöaineista ja eri reaktioista. Tätä on pyritty automatisoimaan eri CASP -menetelmien avulla (Computer-Aided Synthesis Planning).
Ensimmäiset CASP -menetelmät perustuivat heuristisiin algoritmeihin, joissa kemistit käsin koodasivat, miten eri lähtöaineet reagoivat keskenään ja
mikä on reaktion lopputuote. Tämä on kuitenkin osoittautunut toivottomaksi yritykseksi massiivisen datan määrän takia.

Kehitys koneoppimismenetelmissä on kuitenkin tarjonnut uuden lähestymis tavan CASP -menetelmien keshitykseen. Sen sijaan, että
kemistit loisivat heuristisia malleja, niin uudet koneoppimismallit koulutetaan saatavilla olevan datan avulla.
Tämä on todettu merkittävästi enemmän toteutettavaksi lähestymistavaksi.

Koneoppismallien käyttö ja koulutus ei ole kuitenkaan täysin ongelmaton lähestymistapa myöskään.
Ongelmaan liittyen dataa ei välttämättä ole saatavilla ja datan hankkiminen voi olla kallis operaatio.
Tätä varten on kehitetty tietopankkeja, jotka sisältävät massiivisia määriä dataa tietystä aiheesta, esim. Reaxys kemiallisesista raektioista.


\section{Lääkkeen retrosyntetisoinnin haastavuus}

Retorsyntetisoinnin tekee hankalaksi fakta, että yhdiste voidaan muodostaa sadoilla tai tuhansilla eri tavoilla.
Tämä ongelma toistuu rekursiivisesti, kun yhdiste pilkotaan yhdisteisiin, jotka keskenään reagoidessa muodostavat alkuperäisen yhdisteen.
Pienille ja yksinkertaisille yhdisteille tämä vaihtoehto avaruus on pienempi, mutta yhdisteen koon kasvaessa eri tapojen määrä muodostaa haluttu yhdiste kasvaa eksponentiaalisesti.

Tämän takia tarve tätä prosessia yleistäville koneoppimismalleille on suuri.
Miksi lääkkeiden retrosyntetisointi on hankalaa?~\cite{ButlerKeithT2018Mlfm,deAlmeidaA.Filipa2019Socd}

\section{Kehitettyjä apuvälineitä}

\subsection{3N-MCTS}

3N-MCTS on kehitetty koneoppimismalli, joka etsii retrosynteesi polkuja yksinkertaisempiin
ja saatavilla oleviin lähtöaineisiin~\cite{SeglerMarwinHS2018Pcsw}. 3N-MCTS:än kehitti Marwin Seglerin, Mike Preussin ja Mark Wallerin tutkijaryhmä. Kun retrosynteesi polku on varmennettu ja
todettu toimivaksi, niin syötteenä annettu yhdiste on mahdollista syntetisoida laboratoriossa.
3N-MCTS koostuu kolmesta eri koneoppimismallista ja Monte Carlo -puuhaku algoritmista
(\textbf{Monte carlo tree search, MCTS}). Neuroverkot on koulutettu avustamaan puuhaku algoritmia etenemään
fiksuimpaan suuntaan, kun haku algoritmi etsii syntetisointi polkuja ja tarkistamaan, onko ehdotettu reaktio
mahdollinen kyseisellä molekyylille.

Neuroverkot ovat hakupuun laajentumisen suuntaa ohjaava verkko (\textbf{Expansion policy network, EPN}),
MCTS:än rollout toimintoa tukeva Rollout -verkko (\textbf{Rollout policy network, RPN})
ja verkko, joka tarkistaa, onko syntetisointi polku toteutettavissa (\textbf{In-scope filter network, IFN}).

Data, jolla neuroverkot koulutetaan, on peräisin Reaxys -tietokannasta. Reaxyksen omistaa Elsevier kustantamo. Reaxys -tietokannan sisältämä data koostuu
säännöistä, jotka kertovat, mitkä lähtöaineet reagoivat keskenään, mikä reaktio on kysessä ja mikä on reaktion
tuote. Näitä sääntöjä käytetään mallien kouluttamiseen. Reaxys sisältää yli 12.4 miljoonaa sääntöä. Mallien
kouluttamiseen käytetyt säännöt sisältävät vain yksivaiheisia kemiallisia reaktioita ja reaktiossa on mukana vain
yhdestä kolmeen lähtötuotetta. Eri mallien kouluttamiseen käytetiin eri kriteerein suodatettua dataa tietokannasta.

RPN:än kouluttamiseen valittiin datasta vain reaktiossa muuttuneet atomit ja liitokset (reaktiokeskus)
ja lähimmät vierekkäiset atomit. Datasta suodatettiin pois sellaiset reaktiot, jotka ilmaantuivat
alle 50 kertaa ennen vuotta 2015. EPN:än kouluttamiseen valittiin datasta vain reaktiokeskus.
EPN:än datasta suodatettiin pois sellaiset reaktiot, jotka ilmenivät datassa alle kolme kertaa ennen vuotta
2015. Lopulliset reaktio määrät, joilla RPN ja EPN koulutettiin, olivat 17134 ja 301671. Näillä säännöillä
EPN ja RPN koulutetaan toimimaan hakualgoritmia ohjaavina neuroverkkoina.

EPN on toteutettu Highway -neuroverkkona (Highway network, HN). HN on hyvin syvä neuroverkko
tyyppi, joka saattaa jopa sisältää yli sata kerrosta~\cite{VeryDeepNetworks}.

RPN on neuroverkko, jossa on yksi piilotettu taso. RPN koulutettiin samalla tavalla kuin EPN.

IFN on neuroverkko, joka tarkistaa, onko EPN:än ja RPN:än valitsemat reaktio säännöt toteutettavissa.
IFN koulutetaan sekä onnistuneiden että epäonnistuiden reaktioiden avulla. Koska epäonnistuneita
reaktioita ei talleneta tietokantaan, niin kyseinen data generoidaan. Data generoidaan siten, että jos reaktiossa
\[A + B \rightarrow C\] lähtöaineet A ja B muodostavat reaktiossa lopputuotteen C, niin lopputuotteita
D, E, F, jne. ei muodostu (voisi selittää syvemmin). IFN kouluttamista varten luotiin 100 miljoonaa
epäonnistunutta reaktiota ja 10 miljoonaa testaamista varten.

3N-MCTS:ässä IFN ja EPN on yhdistetty toimimaan yhdessä. Tutkittaessa puun tilaa $S_i$ (selitä Si vaihe) jokainen
molekyyli syötetään EPN:älle ja se tulostaa, mitkä reaktiot voivat muodostaa annetun yhdisteen ja näin ollen
myös mitkä lähtöaineet voivat muodostaa annetun yhdisteen. Nämä reaktiot syötetään IFN:älle, joka suodattaa
valituista reaktioista toteutettavissa olevat. Tämän jälkeen algoritmissa iteroidaan neljää vaihetta, jotka
muodostavat lopullisen puun.

(1) Ensimmäisessä vaiheessa algoritmi valitsee seuraavan lupaavimman tilan puusta kunnes puun lehti on saavutettu.
Jos lehdessä käydään ensimmäisen kerran valinta vaiheen aikana, niin lehti arvostellaan simuloimalla hakualgoritmia
$d$ askelta eteenpäin samalla muodostaen synteesi polkua (rollout). Jos lehdessä käydään useamman kuin yhden
kerran valinta vaiheen aikana, niin mahdolliset reaktiot, jotka muodostavat lehden, tutkitaan ja lisätään lehden
lapsiksi (expansion)

(2) Toisessa vaiheessa lupaavien tilojen lapset tutkitaan. Tällöin etsitään lupaavimmat reaktiot, jotka muodostavat
kyseisessä tilassa olevan yhdisteen.

(3) Kolmannessa vaiheessa tarkistetaan lehden tila. Jos lehti on `todistetusti toimiva', niin algoritmi palauttaa
luvun suuremman kuin yksi, jolloin lehteä suositellaan käytettävän synteesipolussa. Muussa tapauksessa lehdelle
suoritetaan rollout, jolloin RPN antaa rekursiivisesti uusia reaktioita niin kauan, kunnes lehti on pilkottu
lähtöaineisiin tai kunnes suurin sallittu syvyys $d$ on saavutettu.

(4) Viimeisessä vaiheessa lehtien arvot päivitetään. Jos lähtöaineet löydetään rolloutin aikana, niin lehti saa
palkonnoksi arvon 1. Jos Kaikkia lähtöaineita ei löydetty, niin lehdelle annetaan osittainen palkinto. Jos yhtään
lähtöainetta ei löytynyt, niin lehti saa arvon -1.

Saatta kuietnkin olla, että synteesi polkua ei voida luoda. Joko synteesin polun tutkimiseen menee liian kauan
aikaa tai synteesi polku sisältää liian monta vaihetta yhdisteen syntetisoimiseen.

\subsection{Expert knowledge aided neural networks}
- expert knowledge aided neural networks~\cite{10.1145/3450439.3451879}

Miten koneoppimista hyödynnetään tällä hetkellä lääkkeiden syntetisoinnissa?~\cite{SeglerMarwinHS2018Pcsw,10.1145/3219819.3219882,10.1145/3450439.3451879}

\chapter{Tulevaisuuden koneoppimisen mallit ja kehitys}

Miten lääkkeiden kehitys tulee hyötymään tulevaisuuden koneoppimisesta?~\cite{ButlerKeithT2018Mlfm}

\cleardoublepage                          %fixes the position of bibliography in bookmarks
\phantomsection
\addcontentsline{toc}{chapter}{\bibname}  % This lines adds the bibliography to the ToC
\printbibliography

\backmatter

\end{document}

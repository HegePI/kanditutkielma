
\documentclass[finnish,twoside,censored,subject,sw-line]{HYthesisML}


\PassOptionsToClass{openright,twoside,a4paper}{report}

\usepackage{csquotes}
\usepackage[style=numeric,bibstyle=numeric,backend=biber,natbib=true,maxbibnames=99,giveninits=true,uniquename=init]{biblatex}

\addbibresource{bibliography.bib}
\DeclareNameAlias{sortname}{family-given}



\usepackage{lmodern}         % Font package, again in some systems.
\usepackage{textcomp}        % Package for special symbols
\usepackage[pdftex]{color, graphicx} % For pdf output and jpg/png graphics
\usepackage{epsfig}
\usepackage{subfigure}
\usepackage[pdftex, plainpages=false]{hyperref} % For hyperlinks and pdf metadata
\usepackage{fancyhdr}        % For nicer page headers
\usepackage{tikz}            % For making vector graphics (hard to learn but powerful)
\usepackage{amsmath, amssymb} % For better math

\singlespacing               %line spacing options; normally use single

\fussy
\title{koneoppimisen menetelmät lääkkeiden/kemiallisessa syntetisoinnin mallennuksessa}

\author{Heikki Pulli}
\date{\today}



\supervisors{}
\examiners{}


\keywords{ulkoasu, tiivistelmä, lähdeluettelo}
\additionalinformation{\translate{\track}}


\begin{document}

% Generate title page.
\maketitle

% Place ToC
\newpage
\mytableofcontents
\mainmatter

\chapter{Johdanto}

Lyhyt johdanto mitä aineessa käsitellään (de novo lääketutkimus ja retrosynteesi)

\chapter{Uusien lääkkeiden löytäminen}

% Miksi uusien lääkkeiden löytäminen on vaikeaa?~\cite{EkinsSean2019Emlf}

Uusien lääkkeiden tuottaminen on pitkä ja kallis prosessi. Tavallisesti aika, jossa
lääke ensin löudetään, testataan ja hyväksytään useiden eri testien läpi, vaihtelee
10-12 vuoden välillä ja hinta on 1-2 miljardin dollarin välillä.~\cite{EkinsSean2019Emlf}.

Lääkefirmat ovatkin alkaneet selvittää, kuinka eri koneoppimisen malleja voidaan
hyödyntää lääketutkimuksessa nopeuttamaan suurimpia pullonkauloja.~\cite{EkinsSean2019Emlf}.
Eri mallit voivat esimerkiksi karsia kaikista harkinnasta olevista lääkkeistä vain
lupaavimmat kandidaatit, joilla on mahdollisuus päästä testeistä läpi tuotantoon.
Tai koneoppimismalleja voidaan hyödyntää täysin uusien lääkeaineiden etsinnässä, joilla
on halutut lääkkeelliset että fysikaaliset ominaisuudet.~\cite{!lähde!}

Ongelman kuvaaminen koneoppimisongelmana~\cite{10.1145/3219819.3219882,KadurinArtur2017dAAG}

\chapter{Kehitettyjä koneoppimismalleja}
- Prototype based drug design~\cite{10.1145/3219819.3219882}

- Deep generative autoencoder~\cite{KadurinArtur2017dAAG}

\chapter{Lääkkeiden syntetisointi ja retrosyntetisointi}

Mitä tarkoittaa lääkkeen syntetisointi?~\cite{deAlmeidaA.Filipa2019Socd}

\chapter{Lääkkeen retrosyntetisoinnin ongelma}

Miksi lääkkeiden retrosyntetisointi on hankalaa?~\cite{ButlerKeithT2018Mlfm,deAlmeidaA.Filipa2019Socd}

\chapter{Koneoppimisen käytön hyödyt}

Miten koneoppiminen voi auttaa uusien lääkkeiden retrosyntetisoinnissa?~\cite{VamathevanJessica2019Aoml}

\chapter{Kehitettyjä apuvälineitä}

3N-MCTS on kehitetty koneoppimismalli, joka etsii retrosynteesi polkuja yksinkertaisempiin
ja saatavilla oleviin lähtöaineisiin~\cite{SeglerMarwinHS2018Pcsw}. Kun retrosynteesi polku on varmennettu ja
todettu toimivaksi, niin syötteenä annettu yhdiste on mahdollista syntetisoida laboratoriossa.
3N-MCTS koostuu kolmesta eri koneoppimismallista ja Monte Carlo -puuhaku algoritmista (\textbf{Monte carlo tree search, MCTS}).
Neuroverkot on koulutettu avustamaan puuhaku algoritmia etenemään fiksuimpaan suuntaan, kun
haku algoritmi etsii syntetisointi polkuja ja tarkistamaan, onko ehdotettu reaktio mahdollinen
kyseisellä molekyylille.

Neuroverkot ovat hakupuun laajentumisen suuntaa ohjaava verkko (\textbf{Expansion policy network, EPN}),
MCTS:än rollout toimintoa tukeva Rollout -verkko (\textbf{Rollout policy network, RPN})
ja verkko, joka tarkistaa, onko syntetisointi polku toteutettavissa (\textbf{In-scope filter network, IFN}).

Data, jolla neuroverkot koulutetaan, on peräisin Reaxys -tietokannasta. Reaxys -tietokanta
sisältää tietoa kemiallisista reaktioista, joita käyettiin mallien kouluttamiseen. Reaxys
sisältää tietoa yli 12.4 miljoonasta kemiallisesta reaktioista ja reaktiot on
kuvattu yksivaiheisina reaktioina. Eri mallien kouluttamiseen käytetiin eri kriteerein
suodatettua dataa tietokannasta.

RPN:än kouluttamiseen valittiin datasta vain reaktiossa muuttuneet atomit ja liitokset (reaktiokeskus)
ja lähimmät vierekkäiset atomit. Datasta suodatettiin pois sellaiset reaktiot, jotka ilmaantuivat
alle 50 kertaa ennen vuotta 2015. EPN:än kouluttamiseen valittiin datasta vain reaktiokeskus.
Datasta suodatettiin pois sellaiset reaktiot, jotka ilmenivät datassa alle kolme kertaa ennen vuotta
2015. Lopulliset reaktio määrät, joilla RPN ja EPN koulutettiin, olivat 17134 ja 301671.

Käytettäviä reaktioita kuvataan sääntöinä. Datasta saadut säännöt liittävät jokaisen reaktion ja
tällöin jokaisen reaktion lopputuotteen reaktiosääntöön. Näillä säännöillä EPn ja RPN koulutetaan
toimimaan hakua ohjaavina neuroverkkoina.

EPN on toteutettu Highway -neuroverkkona (Highway network, HN). HN on hyvin syvä neuroverkko
tyyppi, joka saattaa jopa sisältää yli sata kerrosta~\cite{VeryDeepNetworks}.

RPN on neuroverkko, jossa on yksi piilotettu taso. RPN koulutettiin samalla tavalla kuin EPN.

IFN on neuroverkko, joka tarkistaa, onko EPN:än ja RPN:än valitsemat reaktio säännöt toteutettavissa.
IFN koulutetaan sekä onnistuneiden että epäonnistuiden reaktioiden avulla. Koska epäonnistuneita
reaktioita ei talleneta tietokantaan, niin kyseinen data generoidaan. Data generoidaan siten, että jos reaktiossa
\(A  + B \rightarrow C\) lähtöaineet A ja B reaktiossa muodostavat lopputuotteen C, niin lopputuotteita
D, E, F, jne. ei muodostu (voisi selittää syvemmin). IFN kouluttamista varten luotiin 100 miljoonaa
epäonnistunutta reaktiota ja 10 miljoonaa testaamista varten.

3N-MCTS:ässä IFN ja EPN on yhdistetty prosessiksi. Analyysin vaiheesssa $S_i$ (selitä Si vaihe)jokainen molekyyli syötetään
EPN:älle ja se tulostaa, mitkä reaktiot voivat muodostaa annetun yhdisteen ja näin ollen myös mitkä lähtöaineet
voivat muodostaa annetun yhdisteen. Nämä reaktiot syötetään IFN:älle, joka suodattaa valituista reaktioista
toteutettavissa olevat. Tämän jälkeen iteroidaan neljää vaihetta, jotak muodostavat lopullisen puun.

- expert knowledge aided neural networks~\cite{10.1145/3450439.3451879}

Miten koneoppimista hyödynnetään tällä hetkellä lääkkeiden syntetisoinnissa?~\cite{SeglerMarwinHS2018Pcsw,10.1145/3219819.3219882,10.1145/3450439.3451879}

\chapter{Tulevaisuuden koneoppimisen mallit ja kehitys}

Miten lääkkeiden kehitys tulee hyötymään tulevaisuuden koneoppimisesta?~\cite{ButlerKeithT2018Mlfm}

\cleardoublepage                          %fixes the position of bibliography in bookmarks
\phantomsection
\addcontentsline{toc}{chapter}{\bibname}  % This lines adds the bibliography to the ToC
\printbibliography

\backmatter

\end{document}

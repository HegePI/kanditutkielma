\documentclass[finnish,twoside,censored,essay,sw-line]{HYthesisML}

\PassOptionsToClass{openright,twoside,a4paper}{report}

\usepackage{csquotes}

\usepackage[style=numeric,bibstyle=numeric,backend=biber,natbib=true,maxbibnames=99,giveninits=true,uniquename=init]{biblatex}
\addbibresource{bibliography.bib}

\DeclareNameAlias{sortname}{family-given}

\usepackage{lmodern}         % Font package, again in some systems.
\usepackage{textcomp}        % Package for special symbols
\usepackage[pdftex]{color, graphicx} % For pdf output and jpg/png graphics
\usepackage{epsfig}
\usepackage{subfigure}
\usepackage[pdftex, plainpages=false]{hyperref} % For hyperlinks and pdf metadata
\usepackage{fancyhdr}        % For nicer page headers
\usepackage{tikz}            % For making vector graphics (hard to learn but powerful)
%\usepackage{wrapfig}        % For nice text-wrapping figures (use at own discretion)
\usepackage{amsmath, amssymb} % For better math

\singlespacing               %line spacing options; normally use single

\fussy

\title{Koneoppimisen menetelmät ja käyttökohteet lääketutkimuksessa -ja kehityksessä}

% TM: Contributors to template editors now listed in the beginning of the file in comments
\author{Heikki Pulli}

\date{\today}

\supervisors{}
\examiners{}

% \keywords{ulkoasu, tiivistelmä, lähdeluettelo}
% \additionalinformation{\translate{\track}}

% \classification{\protect{\ \\
% \  General and reference $\rightarrow$ Document types  $\rightarrow$ Surveys and overviews\  \\
% \  Applied computing  $\rightarrow$ Document management and text processing  $\rightarrow$ Document management $\rightarrow$ Text editing
% }}

\begin{document}
\maketitle

\mytableofcontents
\mainmatter

\chapter{Johdanto}

Koneoppimismallien käyttö lääketutkimuksessa ja -kehityksessä on lisääntynyt merkittävästi. Lisäksi tutkimus, jossa selvitetään kuinka eri koneoppimismalleja voidaan hyödyntää lääkekehityksen tarpeisiin on lisääntynyt.
Jotkin organisaatiot järjestävät kilpailuja eri tutkimusyhmien välillä, joiden tarkoituksena on yrittää lötää uusia käyttökohteita koneoppimismalleille lääketutkimuksen tutkimusalueelta.
Yrityksille kannustimena käyttää koneoppimismalleja lääkkeiden kehityksessä toimii mahdollisuus vähentää lääkkeiden käytettyä aikaa ja resursseja, kun koneet voivat antaa omat ennusteensa, millä lääkkeillä on
suurimmat todennäköisyydet päästä kliinisiin testeihin ja niistä läpi tuotantoon. Lisäksi jotkut edistyneet mallit voivat annetusta datasta päätellä, mistä yhdisteistä muodostuu tiettyyn sairauteen tehoava lääke
ja mitkä ovat tämän lääkkeen valmistusvaiheet. Nämä kone voi sitten antaa lääkkeenkehittäjille, jotka voivat validoida koneen antaman yhdisteen toimivuuden ja vaiheiden pätevyyden. Nämä ovat esimerkkejä, kuinka koneoppimista
voidaan hyödyntää, mutta käyttökohteita on useampia. Koneoppimisen hyödyntämisesssä lääketutkimuksessa on kuitenkin vielä monia puutteita ja heikkouksia. Suurin osa  puutteista liittyy datan saatavuuteen ja saatavilla
olevan datan käytettävyyteen. Kaikki saatavilla oleva data ei ole aina tilannekohtaisesti tarpeeksi hyvää, jotta sitä voitaisiin käyttää haluttuun tilanteeseen. Lisäksi hyvän datan tuottamista myös hidastaa sen korkea tuotantokustannus.
Lisäksi eri mallien tulosten validointia vaikeuttaa se, että ei voida koskaan täsyin tietää, miten, kone on tulokseensa päätynyt. Nämä ovat ongelmia, joihin tieteellinen yhteisö yritää saada ratkaisuja~\cite{VamathevanJessica2019Aoml}.
Näistä puutteista huolimatta koneoppiminen on todistanut olevansa tehokas työkalu lääkekehityksessä.

\chapter{Lääketutkimuksessa -ja kehityksessä käytetyt koneoppimisen mallit}

\chapter{Koneoppimismallien käyttökohteet}

\chapter{Koneoppimismallien heikkoudet ja puutteet}

\cleardoublepage
\phantomsection
\addcontentsline{toc}{chapter}{\bibname}
\printbibliography

\backmatter

\end{document}


\documentclass[finnish,twoside,censored,subject,sw-line]{HYthesisML}


\PassOptionsToClass{openright,twoside,a4paper}{report}

\usepackage{csquotes}
\usepackage[style=numeric,bibstyle=numeric,backend=biber,natbib=true,maxbibnames=99,giveninits=true,uniquename=init]{biblatex}

\addbibresource{bibliography.bib}
\DeclareNameAlias{sortname}{family-given}



\usepackage{lmodern}         % Font package, again in some systems.
\usepackage{textcomp}        % Package for special symbols
\usepackage[pdftex]{color, graphicx} % For pdf output and jpg/png graphics
\usepackage{epsfig}
\usepackage{subfigure}
\usepackage[pdftex, plainpages=false]{hyperref} % For hyperlinks and pdf metadata
\usepackage{fancyhdr}        % For nicer page headers
\usepackage{tikz}            % For making vector graphics (hard to learn but powerful)
\usepackage{amsmath, amssymb} % For better math

\singlespacing               %line spacing options; normally use single

\fussy
\title{koneoppimisen menetelmät lääkkeiden/kemiallisessa syntetisoinnin mallennuksessa}

\author{Heikki Pulli}
\date{\today}



\supervisors{}
\examiners{}


\keywords{ulkoasu, tiivistelmä, lähdeluettelo}
\additionalinformation{\translate{\track}}


\begin{document}

% Generate title page.
\maketitle

% Place ToC
\newpage
\mytableofcontents
\mainmatter

\chapter{Johdanto}

Lyhyt johdanto mitä aineessa käsitellään (de novo lääketutkimus ja retrosynteesi)

\chapter{Uusien lääkkeiden löytäminen}

% Miksi uusien lääkkeiden löytäminen on vaikeaa?~\cite{EkinsSean2019Emlf}

Uusien lääkkeiden tuottaminen on pitkä ja kallis prosessi. Tavallisesti aika, jossa
lääke ensin löudetään, testataan ja hyväksytään useiden eri testien läpi, vaihtelee
10-12 vuoden välillä ja hinta on 1-2 miljardin dollarin välillä.~\cite{EkinsSean2019Emlf}.

Lääkefirmat ovatkin alkaneet selvittää, kuinka eri koneoppimisen malleja voidaan
hyödyntää lääketutkimuksessa nopeuttamaan suurimpia pullonkauloja.~\cite{EkinsSean2019Emlf}.
Eri mallit voivat esimerkiksi karsia kaikista harkinnasta olevista lääkkeistä vain
lupaavimmat kandidaatit, joilla on mahdollisuus päästä testeistä läpi tuotantoon.
Tai koneoppimismalleja voidaan hyödyntää täysin uusien lääkeaineiden etsinnässä, joilla
on halutut lääkkeelliset että fysikaaliset ominaisuudet.~\cite{!lähde!}

Ongelman kuvaaminen koneoppimisongelmana~\cite{10.1145/3219819.3219882,KadurinArtur2017dAAG}

\chapter{Kehitettyjä koneoppimismalleja}
- Prototype based drug design~\cite{10.1145/3219819.3219882}

- Deep generative autoencoder~\cite{KadurinArtur2017dAAG}

\chapter{Lääkkeiden syntetisointi ja retrosyntetisointi}

Mitä tarkoittaa lääkkeen syntetisointi?~\cite{deAlmeidaA.Filipa2019Socd}

\chapter{Lääkkeen retrosyntetisoinnin ongelma}

Miksi lääkkeiden retrosyntetisointi on hankalaa?~\cite{ButlerKeithT2018Mlfm,deAlmeidaA.Filipa2019Socd}

\chapter{Koneoppimisen käytön hyödyt}

Miten koneoppiminen voi auttaa uusien lääkkeiden retrosyntetisoinnissa?~\cite{VamathevanJessica2019Aoml}

\chapter{Kehitettyjä apuvälineitä}

- 3N-MCTS~\cite{SeglerMarwinHS2018Pcsw}

3N-MCTS on kehitetty koneoppimismalli, joka etsii retrosynteesi polkuja, kuinka
syötteenä saatu molekyyli voidaan syntetisoida laboratoriossa. 3N-MCTS koostuu
kolmesta eri koneoppimismallista ja Monte Carlo -puuhaku algoritmista. Neuroverkot
on koulutettu avustamaan puuhaku algoritmia etenemään fiksuimpaan suuntaan, kun
haku algoritmi etsii syntetisointi polkuja.

Neuroverkot ovat hakupuun laajentumista ohjaava verkko, Rollout -verkko ja
verkko, joka kertoo, onko syntetisointi polku toteutettavissa.

- expert knowledge aided neural networks~\cite{10.1145/3450439.3451879}

Miten koneoppimista hyödynnetään tällä hetkellä lääkkeiden syntetisoinnissa? ~\cite{SeglerMarwinHS2018Pcsw,10.1145/3219819.3219882,10.1145/3450439.3451879}

\chapter{Tulevaisuuden koneoppimisen mallit ja kehitys}

Miten lääkkeiden kehitys tulee hyötymään tulevaisuuden koneoppimisesta?~\cite{ButlerKeithT2018Mlfm}

\cleardoublepage                          %fixes the position of bibliography in bookmarks
\phantomsection
\addcontentsline{toc}{chapter}{\bibname}  % This lines adds the bibliography to the ToC
\printbibliography

\backmatter

\end{document}

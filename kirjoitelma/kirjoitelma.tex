\documentclass[finnish,twoside,censored,essay,sw-line]{HYthesisML}

\PassOptionsToClass{openright,twoside,a4paper}{report}

\usepackage{csquotes}

\usepackage[style=numeric,bibstyle=numeric,backend=biber,natbib=true,maxbibnames=99,giveninits=true,uniquename=init]{biblatex}
\addbibresource{bibliography.bib}

\DeclareNameAlias{sortname}{family-given}

\usepackage{lmodern}         % Font package, again in some systems.
\usepackage{textcomp}        % Package for special symbols
\usepackage[pdftex]{color, graphicx} % For pdf output and jpg/png graphics
\usepackage{epsfig}
\usepackage{subfigure}
\usepackage[pdftex, plainpages=false]{hyperref} % For hyperlinks and pdf metadata
\usepackage{fancyhdr}        % For nicer page headers
\usepackage{tikz}            % For making vector graphics (hard to learn but powerful)
%\usepackage{wrapfig}        % For nice text-wrapping figures (use at own discretion)
\usepackage{amsmath, amssymb} % For better math

\singlespacing               %line spacing options; normally use single

\fussy

\title{Koneoppimisen menetelmät ja käyttökohteet lääketutkimuksessa -ja kehityksessä}

% TM: Contributors to template editors now listed in the beginning of the file in comments
\author{Heikki Pulli}

\date{\today}

\supervisors{}
\examiners{}

% \keywords{ulkoasu, tiivistelmä, lähdeluettelo}
% \additionalinformation{\translate{\track}}

% \classification{\protect{\ \\
% \  General and reference $\rightarrow$ Document types  $\rightarrow$ Surveys and overviews\  \\
% \  Applied computing  $\rightarrow$ Document management and text processing  $\rightarrow$ Document management $\rightarrow$ Text editing
% }}

\begin{document}
\maketitle

\mytableofcontents
\mainmatter

\chapter{Johdanto}

Koneoppimismallien käyttö lääketutkimuksessa ja -kehityksessä on lisääntynyt merkittävästi. Lisäksi tutkimus, jossa selvitetään kuinka eri koneoppimismalleja voidaan hyödyntää lääkekehityksen tarpeisiin on lisääntynyt.
Jotkin organisaatiot järjestävät kilpailuja eri tutkimusyhmien välillä, joiden tarkoituksena on yrittää lötää uusia käyttökohteita koneoppimismalleille lääketutkimuksen tutkimusalueelta.
Yrityksille kannustimena käyttää koneoppimismalleja lääkkeiden kehityksessä toimii mahdollisuus vähentää lääkkeiden käytettyä aikaa ja resursseja, kun koneet voivat antaa omat ennusteensa, millä lääkkeillä on
suurimmat todennäköisyydet päästä kliinisiin testeihin ja niistä läpi tuotantoon. Lisäksi jotkut edistyneet mallit voivat annetusta datasta päätellä, mistä yhdisteistä muodostuu tiettyyn sairauteen tehoava lääke
ja mitkä ovat tämän lääkkeen valmistusvaiheet. Nämä kone voi sitten antaa lääkkeenkehittäjille, jotka voivat validoida koneen antaman yhdisteen toimivuuden ja vaiheiden pätevyyden. Nämä ovat esimerkkejä, kuinka koneoppimista
voidaan hyödyntää, mutta käyttökohteita on useampia. Koneoppimisen hyödyntämisesssä lääketutkimuksessa on kuitenkin vielä monia puutteita ja heikkouksia. Suurin osa  puutteista liittyy datan saatavuuteen ja saatavilla
olevan datan käytettävyyteen. Kaikki saatavilla oleva data ei ole aina tilannekohtaisesti tarpeeksi hyvää, jotta sitä voitaisiin käyttää haluttuun tilanteeseen. Lisäksi hyvän datan tuottamista myös hidastaa sen korkea tuotantokustannus.
Lisäksi eri mallien tulosten validointia vaikeuttaa se, että ei voida koskaan täsyin tietää, miten, kone on tulokseensa päätynyt. Nämä ovat ongelmia, joihin tieteellinen yhteisö yritää saada ratkaisuja~\cite{VamathevanJessica2019Aoml}.
Näistä puutteista huolimatta koneoppiminen on todistanut olevansa tehokas työkalu lääkekehityksessä.

\chapter{Lääketutkimuksessa -ja kehityksessä käytetyt koneoppimisen mallit}

Lääketutkimuksessa käytetään monia eri koneoppiminen malleja. Eri mallit soveltuvat paremmin eri tilanteisiin,
jolloin tutkija -tai kehittäjäryhmän tulee valita käytettävä malli ongelman mukaisesti. On kuitenkin todettu, että
eri paradigmaa noudattavat mallit sopeutuvat tiettyihin tehtäviin paremmin. On huomattu, että ohjatulla oppimisella
ja vahvitusoppimisella koulutetut mallit soveltuvat paremmin tehtäviin, joissa on tarkoituksena luokitella
dataa tutkittavien ominaisuuksien perusteella. Näitä ovat esimerkiksi kuviin perustuva diagnoosi tai syöpään
vaikuttavien geenien RNAi seulonnassa. Ohjaamatonta koneoppimisen malleja käyetetään puolestaan, kun tutkittavasta datasta halutaan löytää ryppäitä.
Käytettyjä ohjattuja koneoppimismalleja ovat esimerkiksi lineaari regressio ja syvät neuroverkot, jotka soveltuvat
kohteiden luokitteluun. Ohjaamattomia koneoppimismalleja ovat puolestaan k-lähin klusterointi ja itseohjautuva kartta
tai Kohosen kartta.

Kaikki käytettävät mallit ovat kuitenkin vain niin hyviä kuin saatavilla oleva data. Saatavilla olevan datan
tulee olla tarkkaa, tarpeeksi kuvaavaa ja vertaisarvioitua. Kuitenkin tarvittavan datan määrä riippuu käytettävästä
mallista ja tutkittavasta asiasta. Tutkittavasta asiasta riippuen tutkimukseen saattaa liittyä myös käytettävän datan
tuottaminen. Paras data on systemaattisesti tuotettua jossa on ollut mukana mahdollisimman vähän muuttujia
ja joka on kuvaavasti nimetty ja luokiteltu.

\chapter{Koneoppimismallien käyttökohteet}

Lääketutkimuksen tavoitteena on kehittää lääkkeitä, jotka vaikuttavat tautiin muokkaamalla molekyylitason kohdetta.
Koska saatavilla olevaa dataa on enenemissä määrin, joka kuvaa tauteja ja mihin ne vaikuttavat ja mikä niihin
vaikuttaa, niin koneoppimisen menetelmiä voidaan hyödyntää mahdollisten uusien tautien ja niihin tehoavien lääkkeiden
tutkimuksessa.

Yksi koeneoppimisen käyttökohteita on ollutkin taudin aiheuttajan tunnistaminen. Tätä voidaan tutkia
analysoimalla asioita, joihin tauti vaikuttaa. Kun on olemasssa dataa siitä, mitkä mahdolliset asiat vaikuttavat
niihin asioihin joihin tauti vaikuttaa, niin pystytään rajaamaan taudinaiheuttajia.

Koneoppimisen menetelmiä käytetään myös saatavilla olevan lääketieteellisen kirjallisuuden analysointiin.
NLP -menetelmien avulla voidaan seuloa suuresta määrästä tieteellisiä artikkeleita vain ne, jotka ovat
tarpeellisia tilanteeseen. Tämä kuitenkin on riippuvaista siitä, kuinka hyvin julkaisut on annotoitu aiheidensa
mukaan.

Keskeisimmistä koenoppimisen käyttökohteita on ennustaa, kuinka suurella todennäköisyydellä kehitettävä lääke
pääsee kliinisiin testeihin ja niistä läpi. Tutkimuksissa on käytetty dataa, joka kertoo, onko lääke päässyt
testeistä läpi vai ei ja mikä on ollut lääkkeen vaikuttava tekijä. Roullardin tutkimusryhmä päätyi lopputulokseen,
että geeniekspressiivinen data pystyi parhaiten ennustamaan lääkkeen onnistumisen.

Koneoppimista voidaan hyödyntää myös pienten molekyylien vaikutusten ennustamisessa. On huomattu, että
multi-task DNN ovat tähän tarkoitukseen tehokkaampia, kuin aikaisemmin käytetyt menetelmät. Käytetty
One-shot tekniikka tarvitsee vähämmen dataa ja aikaa ennustaakseen tietyn yhdisteen reaktion tietyissä
olosuhteisssa. Multi-task mallit ovat kuitenkin Single-task malleihin verrataen paljon enemmän riippuvaisempia
datasta.

Neuroverkkoja ja nykyisiä puuhaku algoritmeja voidaan myös hyödyntää lääkkeen syntetisoinnin välivaiheiden
suunnittelussa.Tämä voidaan toteuttaa käyttämällä dataa synteettisestä kemiasta. Tämä on kuitenkin vaikea
prosessi, koska tiedon määrä eri reaktioista kasvaa eksponentiaaliesti ja ei aina ole tietoa kaikista
tilanteeseen liittyvisä reaktiosta. Seglerin tutkimusryhmä käytti Monte carlo -puuhakua ja neuroverkkoa apuna
ohjatakseen haku algoritmia oikeaan suuntaan. Tutkimusryhmä käytti dataa Reaxys tietokannasta, joka sisältää
tietoa eri yhdisteistä ja niiden välisistä reaktioista. Tämä menetelmä oli kolmekymmentä kertaa nopeampi kuin
aikaisemmat menetelmät ja tulokset olivat keskivertaisesti yhtä hyvät kuin tutkijan itse tekemä menetelmä
luoda yhdiste synteettisesti.

Uusia biomarkkereita voidaan myös tutkia koenoppimismenetelmillä.

Konenoppimista käytetään myös kuva-analyysissa. Tätät käytetään varsinkin patologisessa kontekstissa, kun
halutaan tutkia esimerkiksi miten jokin tauti ilmenee näytteessä tai miten jokin lääke on vaikuttanut näytteeseen.

\chapter{Koneoppimismallien heikkoudet ja puutteet}

Vaikka koneoppiminen vaikuttaa suoraviivaistavan ja nopeuttavan lääkkeiden tutkimusta ja -kehitystä, niin
malleissa on kuitenkin myös heikkouksia ja puutteita. Yksi näistä on mallien tulosten tulkittavuus ja varmennettavuus.
Koneista nähdään vain, mitä dataa mallille annetaan ja minkä tulksen se antaa käyttäjälle. Miten kone päätyi
lopputulkseensa on liki mahdoton saada selville. Tämä on yksi hidastava aspekti suuremmassa koneoppimisen mallien
käyttöönotossa.

Toinen ongelma on koneiden tulosten toistettavuus. Koneiden antamat tulokset ovat hyvin riippuvaisia koneen
alkuasetuksista ja osittain jopa siitä, missä järjestykssä data koneelle annetaan, vaikka kone saisi täysin
samaa dataa joka tilanteessa.

Hyvän datan puute on myös ongelma. Vaikka dataa olisikin paljon johonkin tilanteeseen, niin datan heikko laatu
saattaa kuitenkin aiheuttaa ongelmia. Koska laadukas data on systemaatiisesti tuotettua, hyvin annotoitua ja
vertaisarvioitua, niin datan luomisessa saattaa kestää hyvinkin kauan aikaa ja datan tuotantokustannus saattaa
nousta hyvinkin suureksi.

Ongelmana on myös tietävän ja osaavan henkilöstön puute. Koska tutkimuksissa käsitellään sekä tietojenkäsittelytieteiden
aiheita että biologian ja lääketutkimuksen aiheita, niin osaavaa henkilöstöä on hankalampi löytää.

\cleardoublepage
\phantomsection
\addcontentsline{toc}{chapter}{\bibname}
\printbibliography

\backmatter

\end{document}

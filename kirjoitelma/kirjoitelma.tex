\documentclass[finnish,twoside,censored,essay,sw-line]{HYthesisML}

\PassOptionsToClass{openright,twoside,a4paper}{report}

\usepackage{csquotes}

\usepackage[style=numeric,bibstyle=numeric,backend=biber,natbib=true,maxbibnames=99,giveninits=true,uniquename=init]{biblatex}
\addbibresource{bibliography.bib}

\DeclareNameAlias{sortname}{family-given}

\usepackage{lmodern}         % Font package, again in some systems.
\usepackage{textcomp}        % Package for special symbols
\usepackage[pdftex]{color, graphicx} % For pdf output and jpg/png graphics
\usepackage{epsfig}
\usepackage{subfigure}
\usepackage[pdftex, plainpages=false]{hyperref} % For hyperlinks and pdf metadata
\usepackage{fancyhdr}        % For nicer page headers
\usepackage{tikz}            % For making vector graphics (hard to learn but powerful)
%\usepackage{wrapfig}        % For nice text-wrapping figures (use at own discretion)
\usepackage{amsmath, amssymb} % For better math

\singlespacing               %line spacing options; normally use single

\fussy

\title{Koneoppimisen menetelmät ja käyttökohteet lääketutkimuksessa -ja kehityksessä}

% TM: Contributors to template editors now listed in the beginning of the file in comments
\author{Heikki Pulli}

\date{\today}

\supervisors{}
\examiners{}

% \keywords{ulkoasu, tiivistelmä, lähdeluettelo}
% \additionalinformation{\translate{\track}}

% \classification{\protect{\ \\
% \  General and reference $\rightarrow$ Document types  $\rightarrow$ Surveys and overviews\  \\
% \  Applied computing  $\rightarrow$ Document management and text processing  $\rightarrow$ Document management $\rightarrow$ Text editing
% }}

\begin{document}
\maketitle

\mytableofcontents
\mainmatter

\chapter{Johdanto}

käyttökohteet
1.  kohteen varmentaminen
2.  Biomerkin tunnistaminen
3.  kliinisistä testeistä kerätyn patologisen datan analyysia

hidasteet/puutteet
1.  tulosten tulkitsemisen vajaavuus
2.  tulosten toistettavuus

tärkeitä nosteita
1.  Korkea laatuisen datan tarve
2.  Koneoppimismallien validointi

koenoppimisesta
1.  Käytetään supervised ja unsuprvised malleja
supervised, kun halutaan luokitella näytteitä ja ennustaa tulevia näytteitä
unsupervised, kun etsitään ryppäitä datasta
2.  Käyttettävä malli tulee valittava käyttökohteen mukaan
Eri malleja vertaillaan usean eri metriikan perusteella, esim. ryhmittely tarkkuus, kappa -arvo,
logaritminen ero...
3. CPU -> GPU -> TPU

Neuroverkoista
Eri neurtoverkkoja ja niiden toiminnalliset erot
1.  Convolutional neural networks (CNN)
2.  Recurrent neural network (RNN)
3.  feedforward network
4.  Deep autoencoder network (DAEN)

Datasta ja datan laadusta ja määrästä
1.  Tulee olla tarkistettua, tarkkaa ja tarpeeksi kuvaavaa
2.  Mallia suunniteltaessa mietittävä, mikä on tarpeeksi dataa
3.  Useat käytännön koneet eivät suoriudukkaan tehtävästään hyvin, koska
niiden koulutukseen tai niiden käsittelemä data vaihtelee laadultaan

Käytännön kohteet lääkkeiden löytmisessä/kehityksessä
1.  Koneoppimista on jo käytetty useassa tutkimuksessa tunnistamaan ja luokittelemaan
dataa näytteistä
2.  NLP -mementelmiä hyödynnetään aineiston etsinnässä. Aineistoa voidaan etsitä
lääke-sairaus, geeni-sairaus ja kohde-lääke konteksteissa
3.  Voidaan käyttää syöpäkohtaisten lääkkeiden etsinnässä. Koulutetaan malli hyödyntäen
kerättyä dataa ja ennustetaan eri lääkkeiden vaikutusta


Koneoppimisen malleja on alettu käyttämään lääketutkimuksen ja -kehityksen tuotantoprosessin eri osa-alueilla ja vaiheissa. Koneoppmisen malleja voidaan hyödyntää esim. eri
solujen tunnistamiseen kuvista, kemiallisten yhdisteiden vaikutusten ennustamista eri soluihin ja parhaimmillaan jopa mallentamaan tiettyyn sairauteen tehoavaa täsmälääkettä~\cite{vamathevan}.
On myös kuitenkin herännyt huoli datasta, jolla koulutetaan koneoppimisen mallit eri tilanteisiin. Huoli on noussut koskien datan laatua, jolla koneet koulutetaan. Tämä myös
koneoppimisen mallien käyttöönotottoa, koska koskaan ei täysin tiedetä, miten koulutettu kone on tulokseensa päätynyt.

\chapter{Lääketutkimuksessa -ja kehityksessä käytetyt koneoppimisen mallit}

\chapter{Koneoppimismallien käyttökohteet}

\chapter{Koneoppimismallien heikkoudet ja puutteet}

\cleardoublepage
\phantomsection
\addcontentsline{toc}{chapter}{\bibname}
\printbibliography

\backmatter

\end{document}

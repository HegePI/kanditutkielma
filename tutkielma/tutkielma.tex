\documentclass[finnish,twoside,censored,tkt,sw-line]{HYthesisML}


% In theses, open new chapters only at right page.
% For other types of documents, may ask "openany" in document.
\PassOptionsToClass{openright,twoside,a4paper}{report}
%\PassOptionsToClass{openany,twoside,a4paper}{report}

\usepackage{csquotes}
%%%%%%%%%%%%%%%%%%%%%%%%%%%%%%%%%%%%%%%%%%%%%%%%%%%%%%%%%
%% REFERENCES
%% Some notes on bibliography usage and options:
%% natbib -> you can use, e.g., \citep{} or \parencite{} for (Einstein, 1905); with APA \cite -> Einstein, 1905 without ()
%% maxcitenames=2 -> only 2 author names in text citations, if more -> et al. is used
%% maxbibnames=99 as no great need to suppress the biliography list in a thesis
%% for more information see biblatex package documentation, e.g., from https://ctan.org/pkg/biblatex

%% Reference style: select one
%% for APA = Harvard style = authoryear -> (Einstein, 1905) use:
% \usepackage[style=authoryear,bibstyle=authoryear,backend=biber,natbib=true,maxnames=99,maxcitenames=2,giveninits=true,uniquename=init]{biblatex}
%% for numeric = Vancouver style -> [1] use:
\usepackage[style=numeric,bibstyle=numeric,backend=biber,natbib=true,maxbibnames=99,giveninits=true,uniquename=init]{biblatex}
%% for alpahbetic -> [Ein05] use:
%\usepackage[style=alphabetic,bibstyle=alphabetic,backend=biber,natbib=true,maxbibnames=99,giveninits=true,uniquename=init]{biblatex}
%

\addbibresource{bibliography.bib}
% in case you want the final delimiter between authors & -> (Einstein & Zweistein, 1905)
% \renewcommand{\finalnamedelim}{ \& }
% List the authors in the Bibilipgraphy as Lastname F, Familyname G,
\DeclareNameAlias{sortname}{family-given}
% remove the punctuation between author names in Bibliography
%\renewcommand{\revsdnamepunct}{ }


%% Block of definitions for fonts and packages for picture management.
%% In some systems, the figure packages may not be happy together.
%% Choose the ones you need.

%\usepackage[utf8]{inputenc} % For UTF8 support, in some systems. Use UTF8 when saving your file.

\usepackage{lmodern}         % Font package, again in some systems.
\usepackage{textcomp}        % Package for special symbols
\usepackage[pdftex]{color, graphicx} % For pdf output and jpg/png graphics
\usepackage{epsfig}
\usepackage{subfigure}
\usepackage[pdftex, plainpages=false]{hyperref} % For hyperlinks and pdf metadata
\usepackage{fancyhdr}        % For nicer page headers
\usepackage{tikz}            % For making vector graphics (hard to learn but powerful)
%\usepackage{wrapfig}        % For nice text-wrapping figures (use at own discretion)
\usepackage{amsmath, amssymb} % For better math
\graphicspath{{../kuvat/}}

\singlespacing               %line spacing options; normally use single

\fussy
%\sloppy                      % sloppy and fussy commands can be used to avoid overlong text lines
% if you want to see which lines are too long or have too little stuff, comment out the following lines
% \overfullrule=1mm
% to see more info in the detailed log about under/overfull boxes...
% \showboxbreadth=50
% \showboxdepth=50



%%%%%%%%%%%%%%%%%%%%%%%%%%%%%%%%%%%%%%%%%%%%%%%%%%%%%%%%%
%% STEP 2:
%%%%%%%%%%%%%%%%%%%%%%%%%%%%%%%%%%%%%%%%%%%%%%%%%%%%%%%%%
%% Set up personal information for the title page and the abstract form.
%% Replace parameters with your information.
\title{Koneoppiminen lääkkeiden kehityksessä}

% TM: Contributors to template editors now listed in the beginning of the file in comments
\author{Heikki Pulli}
\date{\today}



% Set supervisors and examiners, use the titles according to the thesis language
% Prof.
% Dr. or in Finnish toht. or tri or FT, TkT, Ph.D. or in Swedish...
\supervisors{Prof.~D.U.~Mind, Dr.~O.~Why}
\examiners{Prof.~D.U.~Mind, Dr.~O.~Why}


\keywords{ulkoasu, tiivistelmä, lähdeluettelo}
\additionalinformation{\translate{\track}}

%% For seminar reports:
%%\additionalinformation{Name of the seminar}

%% Replace classification terms with the ones that match your work. ACM
%% ACM Digital library provides a taxonomy and a tool for classification
%% in computer science. Use 1-3 paths, and use right arrows between the
%% about three levels in the path; each path requires a new line.

\classification{\protect{\ \\
\  General and reference $\rightarrow$ Document types  $\rightarrow$ Surveys and overviews\  \\
\  Applied computing  $\rightarrow$ Document management and text processing  $\rightarrow$ Document management $\rightarrow$ Text editing
}}

%% if you want to quote someone special. You can comment this line out and there will be nothing on the document.
%\quoting{Bachelor's degrees make pretty good placemats if you get them laminated.}{Jeph Jacques}


%% OPTIONAL STEP: Set up properties and metadata for the pdf file that pdfLaTeX makes.
%% Your name, work title, and keywords are recommended.
\hypersetup{
    unicode=true,           % to show non-Latin characters in Acrobat’s bookmarks
    pdftoolbar=true,        % show Acrobat’s toolbar?
    pdfmenubar=true,        % show Acrobat’s menu?
    pdffitwindow=false,     % window fit to page when opened
    pdfstartview={FitH},    % fits the width of the page to the window
    pdftitle={},            % title
    pdfauthor={},           % author
    pdfsubject={},          % subject of the document
    pdfcreator={},          % creator of the document
    pdfproducer={pdfLaTeX}, % producer of the document
    pdfkeywords={something} {something else}, % list of keywords for
    pdfnewwindow=true,      % links in new window
    colorlinks=true,        % false: boxed links; true: colored links
    linkcolor=black,        % color of internal links
    citecolor=black,        % color of links to bibliography
    filecolor=magenta,      % color of file links
    urlcolor=cyan           % color of external links
}

%%-----------------------------------------------------------------------------------

\begin{document}

% Generate title page.
\maketitle


%%%%%%%%%%%%%%%%%%%%%%%%%%%%%%%%%%%%%%%%%%%%%%%%%%%%%%%%%
%% STEP 3:
%%%%%%%%%%%%%%%%%%%%%%%%%%%%%%%%%%%%%%%%%%%%%%%%%%%%%%%%%
%% Write your abstract to be positioned here.
%% You can make several abstract pages (if you want it in different languages),
%% but you should also then redefine some of the above parameters in the proper
%% language as well, in between the abstract definitions.

\begin{abstract}

    Tämä dokumentti on tarkoitettu Helsingin yliopiston tietojenkäsittelytieteen osaston opin\-näyt\-teiden ja harjoitustöiden ulkoasun ohjeeksi ja mallipohjaksi. Ohje soveltuu kanditutkielmiin, ohjelmistotuotantoprojekteihin, seminaareihin ja maisterintutkielmiin. Tämän ohjeen lisäksi on seurattava niitä ohjeita, jotka opastavat valitsemaan kuhunkin osioon tieteellisesti kiinnostavaa, syvällisesti pohdittua sisältöä.


    Työn aihe luokitellaan
    ACM Computing Classification System (CCS) mukaisesti,
    ks.\ \url{https://www.acm.org/about-acm/class},
    käyttäen komentoa \verb+\classification{}+.
    Käytä muutamaa termipolkua (1--3), jotka alkavat juuritermistä ja joissa polun tarkentuvat luokat erotetaan toisistaan oikealle osoittavalla nuolella.

\end{abstract}

% \begin{otherlanguage}{english}
%     \begin{abstract}
%         Use this otherlanguage environment to write your abstract in another language if needed.

%         Topics are classified according to the ACM Computing Classification System
%         (CCS), see
%         \url{https://www.acm.org/about-acm/class}:
%         check command \verb+\classification{}+. A small set of paths (1--3) should be used, starting from any top nodes
%         referred to bu the root term CCS leading to the leaf nodes. The elements
%         in the path are separated by right arrow, and emphasis of each element individually can be indicated
%         by the use of bold face for high importance or italics for intermediate
%         level. The combination of individual boldface terms may give the reader
%         additional insight.
%     \end{abstract}
% \end{otherlanguage}

% Place ToC
\newpage
\mytableofcontents
\mainmatter

%%%%%%%%%%%%%%%%%%%%%%%%%%%%%%%%%%%%%%%%%%%%%%%%%%%%%%%%%
%% STEP 4: Write the thesis.
%%%%%%%%%%%%%%%%%%%%%%%%%%%%%%%%%%%%%%%%%%%%%%%%%%%%%%%%%
%% Your actual text starts here. You shouldn't mess with the code above the line except
%% to change the parameters. Removing the abstract and ToC commands will mess up stuff.
%%
%% You may wish to include material to avoid browsing the definitions
%% above. Command \include{file} includes the file of name file.tex.
%% As a side effect, subsequent inclusions may force a page break.

% BSc instructions
%\include{bsc_finnish_contents}
%\include{bsc_english_contents}
% MSc instructions
%\include{msc_finnish_contents}
% \include{msc_english_contents}

%%%%%%%%%%%%%%%%%%%%%%%%%%%%%%%%%%%%%%%%%%%%%%%%%%%%%%%%%

\chapter{Johdanto}

Uusien lääkkeiden tuottaminen on pitkä ja kallis prosessi.
Tähän kuuluu useita eri vaiheita ja eri vaiheet vievät eri määrän rahaa ja aikaa.
Nämä ovat sairauden aiheuttajan tunnistaminen, tähän vaikuttavan lääkkeen tunnistaminen, lääkkeen optimointi, lääkkeen ominaisuuksien analysointi ja kliiniset testit.
Näiden jälkeen lääke joko hyväksytään myyntiin tai ei.
Nämä eri vaiheet ovat tavallisesti vieneet 10 - 12 vuotta ja kaikkien eri vaiheiden jälkeen hintaa on muodostunut noin 1 - 3 miljardia dollaria.~\cite{EkinsSean2019Emlf}

Vaiheiden pitkän keston ja suuren hinnan takia tutkijat ja lääkefirmat ovatkin alkaneet tutkia mahdollisia keinoja, jotka nopeuttaisivat tai halventaisivat tätä lääkkeen kehityksen prosessia.
Koneoppimisenmallit ovat nousseet houkuttelevaksi vaihtoehdoksi, joka voisi nopeuttaa tätä prosessia.
Lääkefirmat ovatkin alkaneet selvittää, kuinka eri koneoppimisen malleja voidaan hyödyntää lääketutkimuksessa nopeuttamaan suurimpia pullonkauloja.~\cite{EkinsSean2019Emlf}.
Viimeisen kymmenen vuoden aikana saatavilla olevan laadukkaan datan määrä on kasvanut merkittävästi ja on kehitetty uusia tehokkaampia koneoppimismalleja, joita voidaan hyödyntää lääketutkimuksessa.
Eri mallit voivat esimerkiksi karsia kaikista harkinnasta olevista lääkkeistä vain lupaavimmat kandidaatit, joilla on mahdollisuus päästä testeistä läpi tuotantoon.
Koneoppimismalleja voidaan myös hyödyntää aivan uusien lääkeaineiden etsinnässä.
Tähän tarkoitukseen kehitetyt koneoppimismallit kykenevät etsimään lääkkeitä, joilla on halutut lääkkeelliset että fysikaaliset ominaisuudet.~\cite{VamathevanJessica2019Aoml}
Kehitetyt mallit ovatkin näyttäneet, että koneoppimismallit ovat tehokkaita työkaluja, joita voidaan hyödyntää kaikissa lääketutkimuksen prosessin vaiheissa.

Tässä tektissä paneudutaan syvemmin koneoppimismalleihin, joita käytetään uusien lääkkeiden tunnistaminseen ja näiden tunnistettujen lääkkeiden syntetisoinnin suunnitteluun.

\chapter{Uusien lääkkeiden löytäminen}

% Miksi uusien lääkkeiden löytäminen on vaikeaa?~\cite{EkinsSean2019Emlf}

Yksi ensimmäisistä lääketutkimuksen prosessin osa-alueista on uusien lääkeyhdisteiden löytäminen joko uusiin tai jo tunnettuihin tauteihin.~\cite{EkinsSean2019Emlf}
Tämä on kuitenkin ollut tavallisesti hidas prosessi ja uuden toimivan yhdisteen löytäminen on kestänyt kahdesta kolmeen vuotta.
Lisääntynyt datan määrä on kuitenkin mahdollistanut tämän osa-alueen nopeuttamisen koneoppimismallien avulla.
Tähän ongelmaan on kehitetty useita eri koneoppimismalleja.
Popovan et al. tutkimusryhmä on kehittänyt mallin, joka ehdottaa uutta yhdistettä perustuen mallin syötteenä saamaan ominaisuusvektoriin.~\cite{PopovaMariya2018Drlf}
Shahar Harelin ja Kira Radinskyn tutkijaryhmä puolestaan ovat kehittäneet mallin, joka luo uusia yhdisteitä, jotka perustuvat syötteenä annettuun prototyyppiyhdisteeseen.~\cite{ShaharHarelAndKiraRadinsky}

Jotta koneoppimismalleja voidaan hyödyntää lääketutkimuksessa täytyy olla saatavilla tarpeeksi dataa tutkittavasta aiheesta.~\cite{EkinsSean2019Emlf}
Viimeisimmän kymmenen vuoden aikana saatavilla olevan datan määrä on kasvannut merkittävästi kehitettyjen tietopankkien takia.
Näitä ovat esimerkiksi PubChem ja ChEMBL.

Lääkkeitä myös kehitetään muokkaamalla jo tunnettua lääkeyhdistettä jolla on jo osittain halutut ominaisuudet.~\cite{ShaharHarelAndKiraRadinsky}


\section{Virtuaalinen seulonta}

Kemiallisten yhdisteiden avaruudella tarkoitetaan kaikkien uniikkien yhdisteiden lukumäärää.
Erilaisten kemiallisten yhdisteiden avaruus on suuri.
On arvioitu, että erillaisia kemiallisia yhdisteitä, jotka voivat esiintyä huoneenlämmössä ja nesteessä, voi olla välillä $10^{18} - 10^{180}$.~\cite{SotrifferChristoph2011VSPC}
Yhidsteet, jotka täyttävät lääkkeeltä vaaditut kriteerit, on puolestaan arvioitu olevan noin $10^{60}$.~\cite{SotrifferChristoph2011VSPC}
Tämä itsessään esittää tarpeen tehokkaille algoritmeille ja menetelmille, jotka auttavat karsimaan tästä suuresta määrästä kemiallisia yhdisteitä vain lupaavimmat.

Virtuaalinen seulonta (Virtual Screning, VS) on suosittu lähestymistapa uusien lääkkeiden löytämiseksi.
VS käsittää joukon menetelmiä, joissa tietokoneita hyväksi käyttäen karsitaan kaikesta yhdisteiden avaruudesta vain tietyt kriteerit täyttävät yhdisteet, joita voidaan jatkotutkia ja kehittää lääkkeiksi.
VS menetelmillä tarkoitetaan yleisesti prosesseja, joissa käydään läpi suuria tietokantoja dataa, jotta löydetään haluttu yhdiste.~\cite{SotrifferChristoph2011VSPC}

% \subsection{Koneoppimisen käyttökohteet VS:ässä}

% Ongelman kuvaaminen koneoppimisongelmana~\cite{ShaharHarelAndKiraRadinsky,KadurinArtur2017dAAG}

\section{Kehitettyjä koneoppimismalleja}

% \subsection{Prototyyppiin perustuva lääkkeen suunnittelu}
% - Prototype based drug design~\cite{ShaharHarelAndKiraRadinsky}

\subsection{Controlled molecule generator, CMG}
% - Controlled molecule generator~\cite{ShinBonggun}

CMG (Controlled molecule generator) on koneoppimismalli, joka etsii uusia molekyylejä, jotka perustuvat syötteenä annettuun molekyyliin ja joilla on ennalta määritellyt halutut ominaisuudet.~\cite{ShinBonggun}
Se eroaa muista kehitetyistä malleista siten, että se pystyy optimoimaan annetun molekyylin useampaa ominaisuutta.
Edelliset kehitetyt mallit ovat pystyneet optimoimaan vain yhtä molekyylin ominaisuutta.

CMG:n kehittäjät lähestyvät molekyylin ominaisuuksien optimointiongelmaa merkijonojen käännös/luontiongelmana.
CMG:lle opetetaan, kuinka syötteenä annettu molekyylimerkkijono käännetään molekyylimerkkijonoksi, jolla on lähimpänä haluttuja ominaisuuksia olevat omaisuudet.
CMG tulkitsee ensin annetut merkkijonot hyödyntäen DN:ää (deep network) jonka jälkeen se luo uusia molekyyliyhdisteitä hyödyntäen tätä tulkintaa ja haluttua ominaisuusvektoria.
Koska ominaisuudet annetaan vektorina, niin CMG pystyy optimoimaan useampaa ominaisuutta.
CMG lisäksi hyödyntää ennalta koulutettuja rajoiteverkkoja (constraint network, CN), jolloin vältytään luomasta mahdottomia yhdisteitä.
CMG käyttää näitä verkkoja hyödyksi käyttämällä muokattua Beam Search -algoritmia.

(Taustatietoa muista aiheista/malleista, joita on hyödynnetty CMG:ssä\dots)

CMG perusmalli on samanlainen verrattaen aikaisemmin kehitettyyn The Transformer -malliin (Molecule Translation Network, MTN).\cite{TheTransformer}
CMG:tä on laajennettu tästä lisäämällä siihen tietoa molekyylien ominaisuuksista ja kaksi CN:ää.
Nämä CN:ät ovat ominaisuuksien ennustamiseen tarkoitettu verkko (property prediction network, PPN) ja samanlaisuuksien ennustamiseen tarkoitettu verkko (similarity prediction network, SPN).

MTN eroaa Transformer -mallista kahdella tavalla.
Toisin kuin Transformer, joka käsittelee sanoja ja niistä muodostettuja lauseita, MTN käsittelee yksittäisiä merkkejä ja niistä muodostettuja molekyylejä.
Lisäksi MTN:än piilotettuun kerrokseen on lisätty tietoa kemiallissista ominaisuuksista.
MTN:än kustannus funktio on muotoa \[\mathcal{L} (\theta_T;X,p_x,p_y) = -\frac{1}{N}\frac{1}{M}\sum_{n \in N}\sum_{i \in M}\sum_{v \in V}y_v,j,n \cdot log(\hat{y}_v,j,n).\]

PPN ottaa syötteenä ennustetun molekyylin merkkijonon ($y_i$).
Tämä merkkijono muunnetaan piilotetuiksi vektoreiksi (Parempi termi?) hyödyntäen Long short-term memory (LSTM) -kerrosta.
LSTM muodostaa vektoreita oikealta vasemmalle ja vasemmalta oikealle suunnassa, ja näistä vektoreista valitaan molempien suuntien viimeiset vektorit.
Nämä vektorit yhdistetään ja yhdistevektori syötetään täysin yhdistetylle neuroverkolle.
Tämä verkko sisältää kaksi piilotettua tasoa.

SMP ottaa syötteenä ennustetun molekyylin merkkijono ($y_i$) ja alkuperäisen molekyylin merkkijonon ($x_i$).
Nämä merkkijonot syötetään kahdelle eri LSTM -tasolle, toinen käsittelee alkuperäisen molekyylin merkkijonot ja toinen ennustetun molekyylin merkkijonot.
Nämä LSTM -tasot tomivat samalla periaatteella kuin PPN:ässä ja palauttavat näin ollen neljä eri vektoria.
Nämä vektorit yhdistetään ja tämä yhdiste vektori annetaan täysin yhdistetylle verkolle, joka sisältää kaksi piilotettua tasoa.

Data, jolla, mallit koulutetaan, on peräisin ZINC ja DRD2 dataseteistä. (Kuka hallinnoi?)
Data sisältää kaiken kaikkiaan 257 565 molekyyliä, joista luodaan pareja.
Parit muodostetaan siten, että kahden molekyylin välinen samanlaisuus luku on yli 0.4. (Miten lasketaan?)
Näitä pareja $(X,Y)$ muodostetaan kaiken kaikkiaan 10 827 615 kappaletta.
Lisäksi kaikille molekyyleille lasketaan niiden eri kemiallisten ominaisuuksien arvot, jotka ovat PlogP, QED ja DRD2.
Nämä kuvaavat eri haluttuja ominaisuuksia, joita lääkkeiltä halutaan.

PPN koulutetaan käyttäen 257 565 eri molekyyliä.
nämä molekyylit jaetaan satunnanavaraisella valinnalla koulutus -ja testiryhmiin suhteella 8:2.

SPN kouluttamiseen käytetään osajoukkoa 10 827 615 eri parista.
Tästä määrästä valitaan kymmenen prosentin osajoukko, joka on kaiken kaikkiaan 997 773 molekyylin suuruinen.
Tämä joukko jaettiin samalla tavalla koulutus ja testi ryhmiin kuin PPN:ässä.

\chapter{Uusien lääkkeiden syntetisointi}

Yhdisteen syntetisoinnin suunnittelulla tarkoitetaan prosessia, jossa määritellään, kuinka haluttu yhdiste voidaan tuottaa synteettisesti saatavilla olevista lähtöaineista.~\cite{ColeyConnorW2018MLiC}
Retrosynteesianalyysillä tarkoitetaan puolestaan menetelmää, jonka avulla löydetään halutun yhdisteen tuottamiseen tarvittavat lähtöaineet.
Retrosynteesi toimii siis toiseen suuntaan kuin syntetisointi.
Retrosynteesissä yhdiste pilkotaan rekursiivisesti pienempiin lähtöaineisiin kunnes jäljellä on vain saatavilla olevia lähtöaineita.

Tavallisesti yhdisteen retrosyntetisointi on vaatinut suorittavalta kemistiltä usean vuoden kokemusta ja tietoa saatavilla olevista lähtöaineista ja eri reaktioista.
Tätä on pyritty automatisoimaan eri CASP -menetelmien avulla (Computer-Aided Synthesis Planning).
Ensimmäiset CASP -menetelmät perustuivat heuristisiin algoritmeihin, joissa kemistit käsin koodasivat, miten eri lähtöaineet reagoivat keskenään ja mikä on reaktion lopputuote.
Tämä on kuitenkin osoittautunut toivottomaksi yritykseksi massiivisen datan määrän takia.

Kehitys koneoppimismenetelmissä on kuitenkin tarjonnut uuden lähestymistavan CASP -menetelmien keshitykseen.
Sen sijaan, että kemistit loisivat heuristisia malleja, niin uudet koneoppimismallit koulutetaan saatavilla olevan datan avulla.
Tämä on todettu merkittävästi enemmän toteutettavaksi lähestymistavaksi.

Koneoppismallien käyttö ja koulutus ei ole kuitenkaan täysin ongelmaton lähestymistapa myöskään.
Ongelmaan liittyen dataa ei välttämättä ole saatavilla ja datan hankkiminen voi olla kallis operaatio.
Tätä varten on kehitetty tietopankkeja, jotka sisältävät massiivisia määriä dataa tietystä aiheesta, esim. Reaxys kemiallisesista raektioista.


\section{Lääkkeen retrosyntetisoinnin haastavuus}

Retrosyntetisoinnin tekee hankalaksi fakta, että yhdiste voidaan muodostaa sadoilla tai tuhansilla eri tavoilla.
Tämä ongelma toistuu rekursiivisesti, kun yhdiste pilkotaan yhdisteisiin, jotka keskenään reagoidessa muodostavat alkuperäisen yhdisteen.
Pienille ja yksinkertaisille yhdisteille tämä vaihtoehtoavaruus on pienempi, mutta yhdisteen koon kasvaessa eri tapojen määrä muodostaa haluttu yhdiste kasvaa eksponentiaalisesti.

Tämän takia tarve tätä prosessia yleistäville koneoppimismalleille on suuri.
Miksi lääkkeiden retrosyntetisointi on hankalaa?~\cite{ButlerKeithT2018Mlfm,deAlmeidaA.Filipa2019Socd}
\begin{figure}
    \centering
    \includegraphics[width=\textwidth]{retrosynthesis.jpg}
    \caption{(a) esimerkki reaktio säännöstä, (b) esimerkki mahdollisista reaktioista, kuinka voidaan luoda haluttu yhdiste (keskellä).
        Kuvaa on yksinkertaistettu ja se sisältää vain yksitoista mahdollista reaktiota, jotka tuottavat halutun yhdisteen.
        Väri koodaukset tarkoittavat: Keltainen - kohde yhdiste, Punainen - saatavilla oleva yhdiste, Vihreä - kirjallisuudesta tunnettu yhdiste, liila - tuntematon yhdiste}
    ~\cite{ExpertKnowledgeRetorsynthesis}
\end{figure}

\section{Kehitettyjä apuvälineitä}

\subsection{3N-MCTS}

3N-MCTS on kehitetty koneoppimismalli, joka etsii retrosynteesipolkuja yksinkertaisempiin ja saatavilla oleviin lähtöaineisiin~\cite{SeglerMarwinHS2018Pcsw}.
3N-MCTS:än kehitti Marwin Seglerin, Mike Preussin ja Mark Wallerin tutkijaryhmä.
Kun retrosynteesipolku on varmennettu ja todettu toimivaksi, niin syötteenä annettu yhdiste on mahdollista syntetisoida laboratoriossa.
3N-MCTS koostuu kolmesta eri koneoppimismallista ja Monte Carlo -puuhaku algoritmista (\textbf{Monte carlo tree search, MCTS}).
Neuroverkot on koulutettu avustamaan puuhakualgoritmia etenemään fiksuimpaan suuntaan, kun haku algoritmi etsii syntetisointipolkuja ja tarkistamaan, onko ehdotettu reaktio mahdollinen kyseisellä molekyylille.

Neuroverkot ovat hakupuun laajentumisen suuntaa ohjaava verkko (\textbf{Expansion policy network, EPN}),
MCTS:än rollout toimintoa tukeva Rollout -verkko (\textbf{Rollout policy network, RPN})
ja verkko, joka tarkistaa, onko syntetisointi polku toteutettavissa (\textbf{In-scope filter network, IFN}).

Data, jolla neuroverkot koulutetaan, on peräisin Reaxys -tietokannasta. Reaxyksen omistaa Elsevier kustantamo. Reaxys -tietokannan sisältämä data koostuu
säännöistä, jotka kertovat, mitkä lähtöaineet reagoivat keskenään, mikä reaktio on kysessä ja mikä on reaktion
tuote. Näitä sääntöjä käytetään mallien kouluttamiseen. Reaxys sisältää yli 12.4 miljoonaa sääntöä. Mallien
kouluttamiseen käytetyt säännöt sisältävät vain yksivaiheisia kemiallisia reaktioita ja reaktiossa on mukana vain
yhdestä kolmeen lähtötuotetta. Eri mallien kouluttamiseen käytetiin eri kriteerein suodatettua dataa tietokannasta.

RPN:än kouluttamiseen valittiin datasta vain reaktiossa muuttuneet atomit ja liitokset (reaktiokeskus)
ja lähimmät vierekkäiset atomit. Datasta suodatettiin pois sellaiset reaktiot, jotka ilmaantuivat
alle 50 kertaa ennen vuotta 2015. EPN:än kouluttamiseen valittiin datasta vain reaktiokeskus.
EPN:än datasta suodatettiin pois sellaiset reaktiot, jotka ilmenivät datassa alle kolme kertaa ennen vuotta
2015. Lopulliset reaktiomäärät, joilla RPN ja EPN koulutettiin, olivat 17 134 ja 301 671. Näillä säännöillä
EPN ja RPN koulutetaan toimimaan hakualgoritmia ohjaavina neuroverkkoina.

EPN on toteutettu Highway -neuroverkkona (Highway network, HN). HN on hyvin syvä neuroverkko
tyyppi, joka saattaa jopa sisältää yli sata kerrosta~\cite{VeryDeepNetworks}.

RPN on neuroverkko, jossa on yksi piilotettu kerros. RPN koulutettiin samalla tavalla kuin EPN.

IFN on neuroverkko, joka tarkistaa, onko EPN:än ja RPN:än valitsemat reaktiosäännöt toteutettavissa.
IFN koulutetaan sekä onnistuneiden että epäonnistuiden reaktioiden avulla. Koska epäonnistuneita
reaktioita ei talleneta tietokantaan, niin kyseinen data generoidaan. Data generoidaan siten, että jos reaktiossa
\[A + B \rightarrow C\] lähtöaineet A ja B muodostavat reaktiossa lopputuotteen C, niin lopputuotteita
D, E, F, jne. ei muodostu (voisi selittää syvemmin). IFN kouluttamista varten luotiin 100 miljoonaa
epäonnistunutta reaktiota ja 10 miljoonaa testaamista varten.

3N-MCTS:ässä IFN ja EPN on yhdistetty toimimaan yhdessä. Tutkittaessa puun tilaa $S_i$ (selitä Si vaihe) jokainen
molekyyli syötetään EPN:älle ja se tulostaa, mitkä reaktiot voivat muodostaa annetun yhdisteen ja näin ollen
myös mitkä lähtöaineet voivat muodostaa annetun yhdisteen. Nämä reaktiot syötetään IFN:älle, joka suodattaa
valituista reaktioista toteutettavissa olevat. Tämän jälkeen algoritmissa iteroidaan neljää vaihetta, jotka
muodostavat lopullisen puun.

(1) Ensimmäisessä vaiheessa algoritmi valitsee seuraavan lupaavimman tilan puusta kunnes puun lehti on saavutettu.
Jos lehdessä käydään ensimmäisen kerran valinta vaiheen aikana, niin lehti arvostellaan simuloimalla hakualgoritmia
$d$ askelta eteenpäin samalla muodostaen synteesi polkua (rollout). Jos lehdessä käydään useamman kuin yhden
kerran valintavaiheen aikana, niin mahdolliset reaktiot, jotka muodostavat lehden, tutkitaan ja lisätään lehden
lapsiksi (expansion)

(2) Toisessa vaiheessa lupaavien tilojen lapset tutkitaan. Tällöin etsitään lupaavimmat reaktiot, jotka muodostavat
kyseisessä tilassa olevan yhdisteen.

(3) Kolmannessa vaiheessa tarkistetaan lehden tila. Jos lehti on `todistetusti toimiva', niin algoritmi palauttaa
luvun suuremman kuin yksi, jolloin lehteä suositellaan käytettävän synteesipolussa. Muussa tapauksessa lehdelle
suoritetaan rollout, jolloin RPN antaa rekursiivisesti uusia reaktioita niin kauan, kunnes lehti on pilkottu
lähtöaineisiin tai kunnes suurin sallittu syvyys $d$ on saavutettu.

(4) Viimeisessä vaiheessa lehtien arvot päivitetään. Jos lähtöaineet löydetään rolloutin aikana, niin lehti saa
palkonnoksi arvon 1. Jos Kaikkia lähtöaineita ei löydetty, niin lehdelle annetaan osittainen palkinto. Jos yhtään
lähtöainetta ei löytynyt, niin lehti saa arvon -1.

Saattaa kuitenkin olla, että synteesi polkua ei voida luoda. Joko synteesinpolun tutkimiseen menee liian kauan
aikaa tai synteesi polku sisältää liian monta vaihetta yhdisteen syntetisoimiseen.

\subsection{Expert knowledge aided neural networks}

On myös yleistä, että retrosynteesi polkuja  suunnitelevat mallit hyödyntävät myös heurustiikka.~\cite{ExpertKnowledgeRetorsynthesis}
Badowski et al ryhmä on kehittänyt mallin, joka suunnittelee retrosynteesi polkuja hyödyntämällä neuroverkkoja ja ammattikemistien muodostamia heuristisia malleja suunnitellessaan retrosynteesi polkua.

ICHO:n (Instytut Chemii Organicznej) kouluttamiseen käyettiin dataa 1.4 miljoonasta reaktiota ja niiden lopputuotteista.
Data saadaan julkaisuista artikkeleista ja patenteista.
Reaktiolta kuitenkin vaaditaan, että ainakin yksi reaktio per lopputuote löytyy myös Chematican (taustatietoa) expert-coded säännöstöstä.
Lisäksi datasta suodatettiin pois suojaryhmien reaktiot.

ICHO:sta kehitettiin myös malli, jonka kouluttamisessa käytettiin lisäksi vektoreita, jotka sisäslsivät dataa kemiallisesti intuitiivisistä reaktio piirteistä, joita on käytetty aikaisemmissa tutkimuksissa.
Tätä mallia kutsutaan ICHO+ malliksi, ja sitä käytetään vertailemaan ICHO mallin suoriutumista.

Kehitetyssä ICHO mallissa on kaksi tärkeää seikkaa.
(1) Se laskee reaktioiden ilmaantumistodennäköisyyttä uudelleen perustaen kuinka usein ne ne ilmaantuvat ammattilaisten käyttämissä reaktiossa.
Tämä laskenta tapahtuu mallin koulutusvaiheessa.
Käytännössä tämä tarkoittaa, että koulutusvaiheessa mallilla on tieto, kuinka monta kertaa tietty reaktio sääntö ilmeni datassa ja kuinka monta kertaa kysesitä reaktiota käytettiin luomaan jokin lopputuote.
Malli laskee näiden välisen suhteen, jolloin malli pystyy määrittämään, kuinka usein kyseistä reaktioita kannattaa käyttää.
Jos esimerkiksi jokin reaktio on mainittu datassa kymmenen kertaa ja samassa datassa kyseistä reaktiota käytetään kymmenen kertaa luomaan jokin lopputuote, niin malli käyttää tätä reaktiota, koska se luokitellaan `Helpoksi ja varmaksi toteuttaa'.
Jos suhde taas on pieni, niin reaktio luokitellaan `vaikeaksi toteuttaa'.
(2) Se kykenee antamaan suuremman kuin nolla todennäköisyyden reaktiolle, joka ei ilmaantunut koulutusvaiheessa.
Tämä sen takia, että mallille koulutetaan jatkuva funktio, joka antaa todennäköisyyden jokaiselle reaktio tyypille perustuen, kuinka reaktio muuttaa lopputuotteen lähtötuotteiksi.
Eli jos jokin reaktion sormenjälki (Lisää tietoa, koska useassa mallissa käytetään) tai heuristinen kuvaus on samankaltainen toisen reaktion kanssa, joka ilmenee koulutusdatassa, niin reaktiolle voidaan antaa todennäköisyys sen käytölle.

ICHO mallia vertailtiin aikasiemmin mainitun 3N-MCTS mallin kanssa, jota kustsutaan artikkelissa SW mallina.
Kyseisestä SW mallista kehitettiin myös heuristista dataa hyödyntävä malli SW+.
Lisäksi luotiin SW malli, joka ei käy läpi kaikkia mahdollisia reaktioita, vaan valitsee reaktiot vain niistä reaktioista, jotka eivät aiheuta konfliktia (selitys) ja jotka johtavat vain tiettyyn lopputuotteeseen.
Tätä mallia kutsutaan SW2 malliksi ja täsät luodusta heuristisesta mallista käytetään nimeä SW2+
Lisäksi vertailumalliksi luodaan myös täysin heuristinen malli, joka arvioi reaktioita sen mukaan, kuinka paljon ne yksinkertaistavat lopputuotetta rakenteellisesti.
Tämä malli suosii reaktioita, jotka pilkkovat molekyylin keskeltä, eli puolittavat sen.
Tämä on haluttu lähestymistapa retrosuynteesiin, koska se minimoi syntetisointi vaiheiden määrän.


Miten koneoppimista hyödynnetään tällä hetkellä lääkkeiden syntetisoinnissa?~\cite{SeglerMarwinHS2018Pcsw,ShaharHarelAndKiraRadinsky,ShinBonggun}

\chapter{Lääkkeiden kehitys tulevaisuudessa}

Vaikka koneoppimisen käyttökohteet ovat yleistyneet lääkekehityksessä, niin koneoppimisen lähestymistavaoilla on edelleen potenttiaalia kehittyä.~\cite{ButlerKeithT2018Mlfm}

Yksi kehityksen kohteista on tehokkaampi datan tulkinta pienestä määrästä dataa.
Paikoin kemian ja lääkekehityksen alalla on muihin koneoppimismallien käyttökohteisiin verrattaen vähän dataa saatavilla ja sitä on kallista tuottaa.
Tämä vaatii tutkijoilta enemmän työtä, jotta tieteellisistä julkaisuista saatava data saadaan koneiden hyödyntämään muotoon.
Toinen ratkaisu tähän on meta-oppimis (selitys) lähestymistapa.
Uudet lähestymistavat kuten Neuro Turing kone ja matkija oppiminen mahdollistavatkin oppimisen vähästä määrästä dataa ja Bayes luokittelija on suoriutunut lähes ihmistasoisesti käyttämällä One-shot oppismallia.

Toinen merkittävä edistysaskel on kemiallisten yhdisteiden ja reaktioiden tehokkaampi esitysmalli.
Tähän asti kemiaa on esitetty ihmisten ymmärtämässä muodossa, mutta tämä ei aina ole koneelle paras esitysmalli.
Koneoppimismallit käyttävät tietoa molekyyleistä ja atomeista ja koneoppimismallit ovat niin hyviä kuin nämä kuvaukset.
Hyvältä kuvaukselta vaaditaan, että sen avulla on yksinkertaista hankkia kohteen ominaisuudet ja sen tulee olla mahdollisimman pieni ulottuvuinen.
Uusia kuvausmalleja on kehitetty ja uudet mallit ovat näyttäneet, että ne ovat tehokkaita, niin uusien kuvauksien kehittäminen jatkuu edelleen.

Kemiallista koneoppimista edistävä asia on myös kvanttilaskennan hyväksi käyttö.
Kvanttikoneiden suuri laskuteho pystyy laskemaan uusien mallien koulutusaikaa merkittävästi.

Koenoppimista voidaan myös käyttää tulevaisuudessa mahdollisesti uusien tieteellisten lakien löytämisessä.
Mutta vaikka eri koneoppimisesta kehitetyt mallit ovat ennalta arvattavia, niin ne ei kuitenkaan ole aina tulkittavia.
Neuroverkko voi esimerkiksi oppia ideaalin kaaulain $(pV=nRT)$, mutta neuroverkon kaarien painojen muuttamien ymmärrettäväksi säännöksi ei ole yksinkertainen tehtävä.
Voi olla, että koneoppimismalli pystyy huomaamaan datasta säännön, joka toistuu, mutta jos tutkijat eivät tiedä tai tunne kyseistä sääntöä, niin mallin tulkitseminen on lähes mahdotonta.

\cleardoublepage                          %fixes the position of bibliography in bookmarks
\phantomsection
\addcontentsline{toc}{chapter}{\bibname}  % This lines adds the bibliography to the ToC
\printbibliography

%%%%%%%%%%%%%%%%%%%%%%%%%%%%%%%%%%%%%%%%%%%%%%%%%%%%%%%%%
\backmatter
% \begin{appendices}

%     \appendix{Sample Appendix\label{appendix:model}}
%     usually starts on its own page, with the name and number of the appendix at the top.
%     The appendices here are just models of the table of contents and the presentation. Each appendix
%     Each appendix is paginated separately.

%     In addition to complementing the main document, each appendix is also its own, independent entity.
%     This means that an appendix cannot be just an image or a piece of programming, but the appendix must explain its contents and meaning.

% \end{appendices}
%%%%%%%%%%%%%%%%%%%%%%%%%%%%%%%%%%%%%%%%%%%%%%%%%%%%%%%%%

\end{document}
